%----------------------------------------------------------------------------------------
%	WORK EXPERIENCE SECTION
%----------------------------------------------------------------------------------------
\begin{rSection}{Work Experience}
\begin{rSubsection}{Hatena, Kyoto, Japan}{\em July 2022 - Present}{Software Engineer (part time)}{}
\item Developing a bookmark-based SNS, Hatena bookmark, which is the one of the most popular bookmark websites in Japan.
\item Developing the article classifier that can categorize the attributes of articles.
\end{rSubsection}

\begin{rSubsection}{Apple, Tokyo, Japan}{\em January 2022 - May 2022}{Machine Learning Engineer Internship}{}
\item Developed machine learning models for unsupervised video anomaly detection.
\end{rSubsection}

\begin{rSubsection}{Preferred Networks, Tokyo, Japan}{\em October 2021 - March 2022}{Project Engineer (part time)}{}
\item Worked on a FoodAI project as a part-time researcher.
\item The paper was accepted to CEA++22 [Nishimura+ CEA2022].
\end{rSubsection}

\begin{rSubsection}{Preferred Networks, Tokyo, Japan}{\em August 2021 - September 2021}{Research Internship}{}
\item Joined an internship for mutli-modal FoodAI project.
\end{rSubsection}

\begin{rSubsection}{Kyoto University, Kyoto, Japan}{\em April 2019 - March 2020}{Research Assistant (part time)}{}
\item Created a novel dataset ``r-FG-BB dataset'' which allows a model to ground visual objects with textual descriptions. The paper was accepted to LREC2020 [Nishimura+ LREC2020].
\end{rSubsection}

\begin{rSubsection}{OMRON SINIC X, Tokyo, Japan}{\em February 2020 - March 2020}{Research Internship}{}
\item Proposed a structure-aware procedural text generation model from an image sequence. The paper was accepted at [Nishimura+ IEEE Access21].
\item Contributed to building a dataset for generating a recipe from an instructional video. Joint work with Dr. Atsushi Hashimoto and Yoshitaka Ushiku.
\item Obtained the Japanese patent (patent number: P2021-140551A).
\end{rSubsection}

\begin{rSubsection}{Pixiv, Fukuoka, Japan}{\em November 2017 - March 2019}{Machine Learning Engineer (part time)}{}
\item Developed a neural-network-based perceptual JPEG encoder, which can reduce the average size of images by 15\% without impacting the perceived visual quality of the images.
\item Available on Github: https://github.com/misogil0116/Biscotti
\item Obtained the Japanese patent (patent number: P2020-88740A).
\end{rSubsection}

\begin{rSubsection}{Mercari, OH, USA}{\em November 2017 - November 2017}{Web Application Engineer (part time)}{}
\item Went to U.S to investigate how many people use the app and proposed a new solution to popularise the app.
\end{rSubsection}

\begin{rSubsection}{Wantedly, Tokyo, Japan}{\em August 2017 - September 2017}{Machine Learning Engineer (part time)}{}
\item Developed a named entity recognition (NER) system for the business card management app ``Wantedly People.'' 
\item Improved F1 score by 10\% by replacing a human-crafted machine learning approach with neural-network-based NER tagger of biLSTM-CNN-CRF.
\end{rSubsection}

\end{rSection}